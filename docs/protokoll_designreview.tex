%        File: protokoll.tex
%     Created: Thu Oct 13 10:00 AM 2011 C
% Last Change: Thu Oct 13 10:00 AM 2011 C
%

\documentclass[a4paper]{article}

% german dictionary
\usepackage[ngerman]{babel}

% enconding
\usepackage[utf8]{inputenc}

% graphics
\usepackage{graphicx}

% handle positions
\usepackage{float}

% set borders
\usepackage{a4wide}

% use URLs
\usepackage{url}

% use tables over multiple pages
\usepackage{longtable}

% span a cell over multiple rows
\usepackage{multirow}

% expand table formatting
\usepackage{array}

% header and footer
\usepackage{fancyhdr}
\pagestyle{fancy}
\fancyhf{}

\fancyhead[L]{Design-Review Semesterarbeit}
\fancyhead[R]{Beschlussprotokoll}

\fancyfoot[L]{Andreas Brönnimann}
\fancyfoot[R]{30.01.2012}
\renewcommand{\footrulewidth}{0.5pt}

\begin{document}
\begin{longtable}{p{7cm} p{7cm}}
 \hline
 \multirow{2}{*}{\textbf{Teilnehmer}}  & Dr. Hans-Joachim Böckenhauer\\
         								& Andreas Brönnimann\\
 \hline
	\textbf{Ort} & Zürich, ETH CAB F11\\
 \hline
	\textbf{Datum} & Montag, 30.01.2012\\
\hline
\end{longtable}

\textbf{Beschlüsse}
\begin{itemize}
	\item Aufgabenstellung muss nicht angepasst werden
        \item Für Problemstellungen mit Dimensionen \textgreater 3D können nur selbst erstellte Instanzen verwendet werden
\end{itemize}

\textbf{Weitere Planung}
\begin{itemize}
	\item Beschreibung der Win/Win Strategie
        \item Algorithmen dokumentieren
        \item Weitere TSPLIB-Probleme berechnen
        \item Weitere generierte Probleme berechnen
\end{itemize}
\end{document}
