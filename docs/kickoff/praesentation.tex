\documentclass[12pt]{beamer}
\usepackage[utf8]{inputenc}
\usepackage{lmodern}
\usepackage{german}
\usetheme{Berkeley}
\title[Kick-Off Semesterarbeit]{Implementation einer Win/Win Strategie}
\author{Andreas Brönnimann}
\institute{Hochschule für Technik Zürich}
\date{\today}

\setbeamertemplate{footline}[frame number]
\setbeamertemplate{navigation symbols}{}

\begin{document}

    \begin{frame}
        \titlepage
    \end{frame}

    \begin{frame}
        \frametitle{Ablauf}
        \tableofcontents
    \end{frame}

	\section{Ausgangslage}
	\begin{frame}
		\begin{itemize}
			\item metrisches Travelling Sales Man Problem ($\Delta$TSP)
			\item Win/Win Strategie
			\item Christofides- und den Hoogeveen-Algorithmus
		\end{itemize}
    \end{frame}

	\section{Ziele}
	\begin{frame}
		\begin{itemize}
			\item Vorstellung des TSP
			\item Vorstellung der Win/Win Strategie
			\item Implementation der Win/Win Strategie\footnote{gemäss Paper "`Structural Properties of Hard Metric TSP Inputs"'}
		\end{itemize}
    \end{frame}

	\section{Aufgabenstellung}
	\begin{frame}
		\begin{itemize}
			\item Vorstellung TSP und Win/Win Strategie
			\item Implementation Algorithmus \footnote{gemäss Paper "`Structural Properties of Hard Metric TSP Inputs"'}
			\item Berechnung von Benchmark-Graphen
			\item Berechnung von generierten Graphen
		\end{itemize}
    \end{frame}

	\section{erwartete Resultate}
	\begin{frame}
		\begin{itemize}
			\item Implementation des Algorithmus
			\item Überblick über die Problemstellung
			\item Vorstellung der Win/Win Strategie
			\item Auswertung der Testresultate
		\end{itemize}
    \end{frame}

	\section{Fragen}
	\begin{frame}
		\frametitle{Fragen}
		\begin{figure}[H]
		\centering
			\includegraphics[width=6cm]{gfx/questionmark}
		\end{figure}
	\end{frame}
\end{document}
