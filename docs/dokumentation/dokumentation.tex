%        Fil: scheduling.tex
%     Created: Mon Apr 11 06:00 PM 2011 C
% Last Change: Mon Apr 11 06:00 PM 2011 C
%
\documentclass[a4paper]{article}

% german dictionary
\usepackage[ngerman]{babel}

% enconding
\usepackage[utf8]{inputenc}

% graphics
\usepackage{graphicx}

% handle positions
\usepackage{float}

% set borders
\usepackage{a4wide}

% use URLs
\usepackage{url}

% expand table formatting
\usepackage{array}

% header and footer
\usepackage{fancyhdr}
\pagestyle{fancy}
\fancyhf{}

\fancyhead[L]{Implementation einer Win/Win Strategie für das TSP}
\fancyhead[R]{Semesterarbeit}

\fancyfoot[L]{Andreas Brönnimann}
\fancyfoot[C]{\thepage}
\fancyfoot[R]{\today}
\renewcommand{\footrulewidth}{0.5pt}

% cover
\title {Semesterarbeit\\
Implementation einer Win/Win Strategie für das TSP\\}
\author {Andreas Brönnimann\\
Hochschule für Technik Zürich\\
Dozent: Dr. Hans-Joachim Böckenhauer}
\date {\today}

\begin{document}

% show all references, even the uncited ones
\nocite{*}

% show cover
\maketitle
\setcounter{page}{0}
\thispagestyle{empty}
\newpage

\tableofcontents
\newpage
\section{Einleitung}
\subsection{Ziel und Vorgehensweise}
Ziel dieser Arbeit ist es, die Win/Win Strategie des Papers "`Structural properties of hard metric TSP inputs"'\cite{moemke11} zu Implementieren. Anschliessend wird die Implementation zur Berechnung verschiedener Problemstellungen verwendet. Das Ergebnis wird mit den theoretisch erwarteten Resultaten verglichen und ausgewertet.

Um ein Verständnis für die Implementation zu entwickeln, werden das Traveling Salesman Problem, die Grundlagen von Win/Win Strategien und die verwendeten Algorithmen vorgestellt. Danach werden die wichtigsten Teile der Implementation erklärt, bevor dann die Ergebnisse ausgewertet werden. 

\subsection{Abgrenzung}
\subsection{Motivation}
Mich fasziniert, dass die Problembeschreibung sehr einfach ist, die Lösung jedoch äusserst schwierig.
Ausserdem lassen sich viele Überlegungen mit Papier und Stift aufzeichnen, ohne das ein Computer benötigt wird. 

\newpage
\section{Traveling Salesman Problem}
\subsection{Geschichte}
\subsection{NP-Schwere}
\subsection{Modellierung}
\subsection{Lösungsverfahren}
\newpage
\section{Win/Win Strategie}
\subsection{Grundidee}
\subsection{Anwendung für das Traveling Salesman Problem}
\newpage
\section{Algorithmen}
\subsection{Minimaler Spannbaum - Algorithmus von Prim}
\subsection{Minimales Perfektes Matching - Blossom V}
\subsection{Eulerkreis - Hierholzer}
\subsection{Kürzung Eulerkreis/Eulerpfad zu Hamiltonkreis/Hamiltonpfad}
\newpage
\section{Implementation}
\subsection{Datenstrukturen}
\subsubsection{Graph}
\subsubsection{Knoten}
\subsubsection{Kante}
\subsection{Minimaler Spannbaum}
\subsubsection{Eulerkreis}

\newpage

\section{Berechnungen}
\subsection{TSPLIB}
\subsection{Zufällige Graphen}

\newpage

% bibliography
\bibliographystyle{plain}
\bibliography{bibliographie}
\end{document}
