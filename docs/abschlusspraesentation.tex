\documentclass[12pt]{beamer}
\usepackage[utf8]{inputenc}
\usepackage{lmodern}
\usepackage{german}
\usetheme{Berkeley}
\title[Semesterarbeit]{Implementation einer Win/Win Strategie fur das Traveling Salesman Problem}
\author{Andreas Brönnimann}
\institute{ZHAW - Zürcher Hochschule für Angewandte Wissenschaften}
\date{11.04.2012}

\setbeamerfont{footnote}{size=\tiny}
\setbeamertemplate{footline}[frame number]
\setbeamertemplate{navigation symbols}{}

\begin{document}

    \begin{frame}
        \titlepage
    \end{frame}

    \begin{frame}
        \frametitle{Ablauf}
        \tableofcontents
    \end{frame}

    \section{Aufgabenstellung}
    \begin{frame}
        \frametitle{Aufgabenstellung}
	    \begin{itemize}
                \item Aufgabenstellung
            \end{itemize}
    \end{frame}
    \section{Traveling Salesman Problem}
    \begin{frame}
        \frametitle{Traveling Salesman Problem}
	    \begin{itemize}
                \item Einleitung zum TSP
            \end{itemize}
    \end{frame}
    \begin{frame}
        \frametitle{Traveling Salesman Problem - Definition}
	    \begin{itemize}
                \item Definition
            \end{itemize}
    \end{frame}
    \begin{frame}
        \frametitle{Traveling Salesman Problem - Geschichte}
	    \begin{itemize}
                \item Geschichte
            \end{itemize}
    \end{frame}
    \begin{frame}
        \frametitle{Traveling Salesman Problem - Lösungsverfahren}
	    \begin{itemize}
                \item Lösungsverfahren
            \end{itemize}
    \end{frame}
    \section{Win/Win Strategie}
    \begin{frame}
        \frametitle{Win/Win Strategie - Grundidee}
	    \begin{itemize}
                \item Win/Win Strategie
            \end{itemize}
    \end{frame}
    \begin{frame}
        \frametitle{Win/Win Strategie - Anwendung TSP}
	    \begin{itemize}
                \item Win/Win Strategie
            \end{itemize}
    \end{frame}
    \section{Algorithmus}
    \begin{frame}
    \frametitle{Algorithmus}
        \begin{scriptsize}
        \textbf{Eingabe:} Ein kompletter Graph $G = (V,E)$, eine metrische Kostenfunktion $c: E \rightarrow \mathbb{Q}^+$ und zwei Knoten $s$ und $t$.

        \begin{itemize}
            \item[1.] Minimalen Spannbaum $T$ von $G$ berechnen.
            \item[2.] Minimales Perfect Matching\footnote{für das Perfect Matching wird eine bestehende Implementation verwendet} $M_C$ der ungeraden Knoten des minimalen Spannbaumes $T$ von $G$ berechnen.
            \item[3.] Minimales Perfect Matching $M_P$ der ungeraden Knoten des Multigraphen $T$ + \{$s$, $t$\} von $G$ berechnen.
            \item[4.] Die Eulertour Eul$_C$ des Multigraphen T $\cup$ $M_C$ und den Eulerpfad Eul$_P$ des Multigraphen T $\cup$ $M_P$ berechnen.
            \item[6.] Eul$_C$ zu einer Hamiltontour $H_C$ und Eul$_P$ zu einem Hamiltonpfad $H_P$ kürzen.
        \end{itemize}
        \textbf{Ausgabe:} $H_C$ and $H_P$.
        \end{scriptsize}
    \end{frame}

    \section{Berechnungen}
    \begin{frame}
        \frametitle{Berechnungen}
	    \begin{itemize}
                \item Berechnungen
            \end{itemize}
    \end{frame}

    \section{Fragen}
    \begin{frame}
    \frametitle{Fragen}
        \begin{figure}[H]
	    \centering
	        \includegraphics[width=6cm]{gfx/questionmark}
        \end{figure}
    \end{frame}
\end{document}
